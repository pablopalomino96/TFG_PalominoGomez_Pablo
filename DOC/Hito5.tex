Una vez implementada la funcionalidad de simulación, debe pensarse en añadir una persistencia al proyecto de los elementos que intervienen. Esto es necesario y primordial si se pretende crear una aplicación web para realizar simulaciones. Puesto que se desarrollará una API usando Flask~\cite{Flask} que hará la función de servidor, se ha dedicido utilizar la herramienta para gestión de base de datos SQLAlchemy.
\subsection{Persistencia con SLQAlchemy}
SQLAlchemy~\cite{SqlAl} proporciona un kit de herramientas SQL que permiten manejar bases de datos de manera eficiente. Está formado por dos componentes:
\begin{itemize}
\item \textit{Core}: Es un conjunto de herramientas de SQL que da lugar a un nivel de abstracción sobre el mismo, mediante un lenguaje que utiliza expresiones generativas en Python para expresar órdenes SQL.
\item \textit{ORM}: Se trata de un asignador relacional de objetos, es decir, permite crear una base de datos de objetos virtuales que permite manipular la información de la base de datos, a priori incompatible, como objetos utilizable por un lenguaje de programación orientada a objetos.
\end{itemize}
Mediante las consultas basadas en funciones permite ejecutar las cláusulas SQL a través de funciones y expresiones en Python. Se pueden realizar numerosas acciones como subconsultas seleccionables, insertar, actualizar, eliminar o declarar un objeto, combinaciones internas y externas sin necesidad de utilizar lenguaje SQL. El ORM permite almacenar en caché las colecciones y referencias de objetos una vez han sido cargados, dando lugar a que no sea necesario emitir SQL en cada acceso.\\

SQLAlchemy puede trabajar con bases de datos de SQLite, Postgresql, MySQL, Oracle, MS-SQL, Sybase y Firebird, entre otros.
\subsection{Modelos User y Home}
Se deben determinar los datos que se van a persistir en el sistema. Puesto que se tratará de una aplicación web con la que interactuarán usuarios, resulta interesante almacenarlos. Cada uno de estos usuarios realizará simulaciones del sistema. Como se vió anteriormente, cuando se instancia un objeto de la clase Simulation, el constructor de la misma recibe varios argumentos necesarios para llevar a cabo la simulación, de los cuales la mayoría son información acerca del hogar en el que se realizará la simulación (número de módulos fotovoltaicos, código de la ciudad del hogar, etc). La persistencia en base de datos de la información del hogar de cada usuario mejoraría esta situación, pues la mayoría de la información que necesita la clase Simulation sería proporcionada del hogar almacenado en base de datos de ese usuario. Por lo tanto, son necesarias dos tablas en la base de datos: \textit{Users} y \textit{Homes}, entre las cuáles existe una relación \textit{one to one}. Este tipo de relación SQL hace que ambas tablas tengan un atributo de referencia a la otra que se conoce como \textit{\textbf{foreign key}}, la cuál lo convierte en una relación bidireccional. Cada \textit{User} tendrá un \textit{Home} y viceversa. Para crear las tablas y mostrar la estructura lógica de cada una, así como sus limitaciones y atributos, se debe crear un \textbf{modelo de base de datos}.
\begin{itemize}
\item \textbf{Modelo User}
\item \textbf{Modelo Home}
\end{itemize}
\subsection{Creación de un servidor con Flask}
\subsection{Desarrollo de la interfaz web}
