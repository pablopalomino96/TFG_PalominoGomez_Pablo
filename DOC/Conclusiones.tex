\chapter{Conclusiones}
\label{cap:Conclusiones}
En este capítulo se muestran las conclusiones obtenidas tras el desarrollo de este \gls{TFG} junto con posible desarrollo futuro a realizar que permitiría ampliar la funcionalidad del sistema.\\
\section{Satisfacción de objetivos}
El objetivo principal del \gls{TFG} ha sido la \textbf{construcción de un sistema inteligente para la simulación de la distribución óptima de energía entre elementos generadores y elementos cosumidores}. Dicho objetivo se ha completado satisfactoriamente durante el desarrollo. Se ha creado un sistema capaz de minimizar el coste energético de una instalación representada por un conjunto de variables, restricciones y condiciones climáticas en un día determinado.\\

El objetivo parcial referente a la \textbf{identificación y adquisición de los datos y variables que definen el sistema} se ha completado satisfactoriamente durante las iteraciones 1 y 2 del desarrollo (Secciones~\ref{sec:hito1} y~\ref{sec:hito2}), donde se trabajó en la obtención de las variables requeridas (precios de cada una de las energías, situación meteorológica, consumo, etc) y en su adaptación al sistema (transformación de los estados meteorológicos a valores discretos, tratado de cada una de las APIs utilizadas, etc).\\

Otro objetivo parcial era \textbf{establecer las relaciones entre las variables y restricciones que determinan el sistema}. Este objetivo se ha completado satisfactoriamente mediante el resultado obtenido de la iteración 3 (Sección~\ref{sec:hito3}) del desarrollo. En dicha iteración se trabajó en la definición del \gls{PSR} planteado, determinando los dominios de las variables que lo componen e identificando las restricciones relativas a la configuración de la instalación, propiedades físicas de generación y consumo de energía y condiciones climáticas. También se determina la función objetivo que dará solución al \gls{PSR} minimizando el coste energético.\\

El siguiente objetivo parcial pasaba por \textbf{implementar la generación optimizada de energía teniendo en cuenta las relaciones y restricciones existentes}. Hace referencia al núcleo del sistema, pues la satisfacción de este objetivo permitiría generar la optimización deseada. Este objetivo ha sido completado con éxito en la iteración 4 (Sección~\ref{sec:hito4}) haciendo uso de algoritmos de programación lineal que permiten minimizar la función objetivo generando una distribución óptima de energía.\\

El último objetivo parcial requerido era \textbf{hacer usable el sistema para realizar simulaciones a demanda}. La iteración 5 (\ref{sec:hito5}) del desarrollo consistió en la creación de una aplicación web que permitiera interactuar a los usuarios con el sistema, configurando las características de su hogar y pudiendo realizar simulaciones de un día concreto. Gracias a los resultados de estas simulaciones, los usuarios podrían comparar la distribución energética y gasto económico que se hubiera producido empleando el sistema propuesto frente a lo ocurrido realmente con el consumo exclusivo de la red eléctrica. Este objetivo se cumple con éxito pues se dispone de una aplicación totalmente funcional que cuenta con interfaz, lógica y persistencia encargada de realizar lo definido.\\

Finalmente, se definió un objetivo parcial para la \textbf{integración del sistema en la nube} obteniendo ubicuidad de acceso y desvinculación de la máquina local. Durante la iteración 6~\ref{sec:hito6} se realizó la migración a la nube, por un lado el servidor se integró en los servicios de Amazon Web Services (AWS) y por otro lado la base de datos fue integrada en los servicios ofrecidos por IBM Cloud. Esto demuestra que el objetivo se ha cumplido con éxito.\\
\section{Ámbito profesional}
En el ámbito profesional se han obtenido beneficios académicos al ampliar conocimientos acerca de sistemas inteligentes, uso de APIs de servicios, tecnologías de computación en la nube y algoritmos de optimización y desarrollo. Se han obtenido conocimientos en lo relativo a desarrollo de \textit{frontend}, marco donde el alumno no había trabajado anteriormente siendo necesario haber adquirido unas competencias con las que no se contaban antes de la realización de este trabajo.\\
Se ha estudiado y utilizado una metodología ágil de desarrollo software como es \textbf{programación extrema (XP)} que ha permitido al alumno obtener conocimientos acerca del uso de metodologías ágiles, elección de la más adecuada para un caso concreto y su aplicación, algo que será de verdadera utilidad y que constituye una base en su futuro profesional junto con los conocimientos adquiridos durante el grado. El haberse enfrentado al desarrollo de un proyecto software en un tiempo limitado, partiendo desde cero y dando como resultado un \textbf{sistema inteligente} usable, mantenible y mejorable ha permitido al alumno llevar a la práctica cada uno de los conceptos y habilidades obtenidas tras la realización del Grado en Ingeniería Informática.\\

Se ham ampliado y llevado a la práctica numerosos conocimientos en el software de control de versiones Git. El repositorio de este \gls{TFG} se ha estructurado en dos bloques: SOURCE y DOC, los cuáles contienen el software del sistema y la documentación del proyecto, respectivamente. Cada uno de ellos ha sido desarrollado paralelamente en dos ramas distintas y posteriormente mezcladas en la rama \textit{master}. En el caso del desarrollo, se ha creado una nueva rama partiendo de la propia \textit{source} por cada una de las iteraciones \gls{XP}, siendo mezcladas de nuevo en \textit{source} tras pasar la revisión del director. Para ello se han creado \textit{pull requests} por cada iteración, en cuyas descripciones se indicaban los puntos e historias de usuario desarrollados en la iteración, obtenidos del tablero Kanban de la misma.\\

Puesto que se han empleado herramientas de \textit{time tracking} típicas en el desarrollo ágil del software, se ha obtenido como resultado un total de 305 horas de trabajo imputadas. De esta cantidad de tiempo, un total de 169 horas corresponde a desarrollo software; 120 horas corresponden a documentación; y 16 horas han sido empleadas en reuniones de planificación, revisión y consulta de dudas con los directores.
\section{Trabajo futuro}
En cuanto al posible trabajo futuro se exponen a continuación mejoras que no se han podido implementar debido a la limitación de tiempo en la realización de un \gls{TFG}.
\begin{itemize}
\item \textbf{Alamacenar en base de datos las simulaciones realizadas}. El sistema realiza simulaciones cuyos resultados muestra y permite descargar al usuario, pero no se realiza una persistencia de los mismos por usuario. Resultaría interesante almacenar las simulaciones que realiza un usuario.
\item \textbf{Aplicar \textit{machine learning} a la base de datos para obtener conocimiento acerca del consumo eléctrico}. Mediante las simulaciones almacenadas comentadas anteriormente, se podrían aplicar algoritmos de aprendizaje automático a la base de datos cuando se cuente con un volumen representativo de datos pudiendo dar lugar a conocimiento.
\item \textbf{Aumentar la precisión de obtención de variables para generar resultados más concretos}. El valor de energía fotovoltaica es calculado en función del estado meteorológico actual y el número de módulos fotovoltaicos con los que se cuenta. Esto puede ser mucho mas preciso pues hay más factores dependientes como pueden ser el grado de incidencia de la luz solar en el módulo, viento, sensación térmica o desgaste de la placa que no se han podido tener en cuenta debido a la limitación temporal.
\item \textbf{Trabajar en la mejora y ampliación de la página web}. Puesto que el tiempo ha sido limitado han quedado pendientes posibles mejoras a realizar en la página web como son mayor control de sesiones, funcionalidad \textit{responsive} (adecuación al formato del dispositivo), con la que se cuenta de manera limitada pues algunos elementos se solapan si se accede desde determinados dispositivos móviles. Otra ampliación de la web si se implementa la persistencia de simulaciones sería añadir una nueva sección que permitiera consultar estadísticas y reportes de simulaciones realizadas por el usuario.
\item \textbf{Implementación de un bot que mediante \textit{web scraping} obtenga el consumo del usuario} en lugar de ser obtenido por medio de la carga de un fichero de texto por parte del cliente en la web. El usuario únicamente añadiría su cuenta de Endesa (con previa aceptación de permisos) y el bot se encargaría de obtener la información de su consumo para realizar simulaciones, teniendo únicamente que elegir el día de simulación.
\end{itemize}
