\chapter{Conclusiones}
\label{cap:Conclusiones}
En este capítulo se muestran las conclusiones obtenidas tras el desarrollo de este trabajo fin de grado junto con posible desarrollo futuro a realizar que permitiría ampliar la funcionalidad del sistema obtenido.\\
\section{Satisfacción de objetivos}
El objetivo principal del trabajo fin de grado ha sido la \textbf{construcción de un sistema inteligente para la simulación de la distribución óptima de energía entre elementos generadores y elementos cosumidores}. Dicho objetivo se ha completado satisfactoriamente durante el desarrollo. En función de unos valores que definen un estado por horas referentes a un día concreto, el sistema realiza una optimización para distribuir la energia entre ellos con el menor gasto económico posible.\\

El objetivo parcial referente a \textbf{identificar e implementar medios de adquisición de los datos y variables que definen el sistema} se ha completado satisfactoriamente durante las iteraciones 1 y 2 del desarrollo, donde se trabajó en la obtención de las variables requeridas (precios de cada una de las energías, situación meteorológica, consumo, etc) y en su adaptación al sistema (transformación de los estados meteorológicos a valores discretos, tratado de cada una de las APIs utilizadas, etc).\\

Otro objetivo parcial era \textbf{establecer las relaciones y restricciones propias del modelo}. Este objetivo se ha completado satisfactoriamente mediante el resultado obtenido de la iteración 3 del desarrollo. En dicha iteración se trabajó en lo relativo al problema de satisfacción de restricciones planteado (variables que lo componen y sus dominios, restricciones a tener en cuenta, etc) y además se implementó la clase que hace referencia al modelo de simulación.\\

El siguiente objetivo parcial pasaba por \textbf{añadir una inteligencia artificial para la generación optimizada de energía y dar lugar a una planificación}. Hace referencia al núcleo del sistema, pues la satisfacción de este objetivo permitiría generar la optimización deseada. Este objetivo ha sido completado con éxito en la iteración 4 haciendo uso de algoritmos de programación lineal que han permitido generar una optimización y dar lugar a la planificación esperada, desglosando el resultado en bruto obtenido.\\

El último objetivo parcial requerido era \textbf{hacer usable el sistema para realizar simualciones a demanda}. Para ello la iteración 5 del desarrollo consisitió en la creación de una aplicación web que permitiera a usuarios registrados realizar simulaciones de consumo óptimo de los días que deseen. Este objetivo se cumple con éxito pues se dispone de una aplicación totalmente funcional que cuenta con interfaz, lógica y persistencia encargada de realizar lo definido.\\

Finalmente, se definió un objetivo parcial para \textbf{integración del sistema en el \textit{cloud computing}} obteniendo ubicuidad de acceso y desvinculación de la máquina local. Durante la iteración 6 se realizó la migración a la nube, por un lado el servidor se integró en los servicios de Amazon Web Services (AWS) y por otro lado la base de datos fue integrada en los servicios ofrecidos por IBM Cloud. Esto muestra que el objetivo se ha cumplido con éxito.\\
\section{Ámbito profesional}
En el ámbito profesional se han obtenido beneficios académicos sobre sistemas inteligentes, uso de APIs de servicios, tecnologías de computación en la nube y algoritmos de optimización y desarrollo. Se han obtenido conocimientos en lo relativo a desarrollo de \textit{frontend}, marco donde el alumno no había trabajado anteriormente siendo necesario haber adquirido unas competencias con las que no se contaban antes de la realización de este trabajo.\\
Se ha estudiado y llevado a cabo una metodología ágil de desarrollo software como es \textbf{programación extrema (XP)} que ha permitido al alumno obtener conocimientos acerca del uso de metodologías ágiles, elección de la más adecuada para un caso concreto y su aplicación, algo que será de verdadera utilidad y que constituye una base en su futuro profesional junto con los conocimientos adquiridos durante el grado.\\

El haberse enfrentado al desarrollo de un proyecto software en un tiempo limitado partiendo desde cero y dando como resultado un \textbf{sistema inteligente} usable, mantenible y mejorable ha permitido al alumno llevar a la práctica cada uno de los conceptos y habilidades obtenidas tras la realización del Grado en Ingeniería Informática.
\section{Trabajo futuro}
En cuanto al posible trabajo futuro se exponen a continuación mejoras que no se han podido implementar debido a la limitación de tiempo en la realización de un trabajo fin de grado.
\begin{itemize}
\item \textbf{Persistir las simulaciones realizadas}. El sistema realiza simulaciones cuyos resultados muestra y permite descargar al usuario, pero no se realiza una persistencia de los mismos por usuario. Resultaría interesante almacenar las simulaciones que realiza un usuario para realizar la siguiente posible ampliación.
\item \textbf{Aplicar \textit{machine learning} a la base de datos para obtener conocimiento acerca del consumo eléctrico}. Mediante las simulaciones persistidas comentadas anteriormente, se podrían aplicar algoritmos de aprendizaje automático a la base de datos cuando se cuente con un volumen representativo de datos dando lugar a conocimiento.
\item \textbf{Aumentar la precisión de obtención de variables para generar resultados más concretos}. El valor de energía fotovoltaica es calculado en función del estado meteorológico actual y el número de módulos fotovoltaicos con los que se cuenta. Esto puede ser mucho mas preciso pues hay más factores dependientes como pueden ser el grado de incidencia de la luz solar en el módulo, viento, sensación térmica o desgaste de la placa que no se han podido tener en cuenta debido a la limitación temporal.
\item \textbf{Trabajar en la mejora y ampliación de la página web}. Puesto que el tiempo ha sido limitado han quedado pendientes posibles mejoras a realizar en la página web como son mayor control de sesiones, funcionalidad \textit{responsive} (adecuación al formato del dispositivo), con la cuál cuenta vagamente pues algunos elementos se solapan si se accede desde determinados dispositivos móviles. Otra ampliación de la web si se implementa la persistencia de simulaciones sería añadir una nueva sección que permitiera consultar estadísticas y reportes de simulaciones realizadas por el usuario.
\item \textbf{Implementación de un bot que mediante \textit{web scraping} obtenga el consumo del usuario} en lugar de ser obtenido por medio de la carga de un fichero de texto por parte del cliente en la web. El usuario únicamente añadiría su cuenta de Endesa (con previa aceptación de permisos) y el bot se encargaría de obtener la información necesaria de su consumo para realizar simulaciones, teniendo el usuario únicamente que elegir el día de simulación.
\end{itemize}
