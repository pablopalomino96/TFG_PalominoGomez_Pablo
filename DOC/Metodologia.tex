\chapter{Metodología}
\label{cap:Metodologia}

En el proyecto se emplea como metodología software un modelo iterativo e incremental,
que da lugar a una toma de decisiones a corto plazo lo que se traduce en ampliar requisitos y
soluciones en cada iteración (sprint) en función de las necesidades. Esto proporciona inmediatez
y funcionalidad en el proyecto lo que hace que exista una mayor motivación e implicación en
el mismo. Además, permite encontrar y solucionar errores a lo largo del trabajo y hace que el
cliente esté mas implicado debido a las numerosas entregas a lo largo del desarrollo del trabajo.

Como metodología de desarrollo y gestión de proyecto se usará \textbf{Extreme Programming (XP)}.\\
La programación extrema~\cite{Newk02} es un método ligero de desarrollo iterativo e incremental formulado por Kent Beck. Consta de varios períodos:
\begin{description}
	\item  \textbf{Exploración:}
	Período donde el objetivo será identificar, priorizar y estimar los requisitos del trabajo, por lo tanto se obtendrá como salida un documento de especificación de requisitos. El cliente expone sus necesidades y los programadores deben eliminar la ambigüedad para asegurarse de que los objetivos pueden ser alcanzados. 
	
	\item \textbf{Punto de Fijación:}
	Se trata de una prueba rápida para profundizar en un determinado aspecto. Este punto se puede concretar durante la exploración o en cualquier otro momento en el que el equipo necesite resolver una cuestión.
	
	\item \textbf{Planificación de la Versión:}
	Cada versión del sistema proporciona un valor de negocio al cliente, quien, en cada planificación de versión, selecciona las historias o requisitos que van a ser implementados. Esto proporciona el máximo valor de negocio aunque no sea lo más acertado técnicamente.
	
	\item \textbf{Planificación de la Iteración:}
	Cada versión se divide en varias iteraciones. La longitud de iteración del trabajo se decide al principio y se mantiene constante durante el desarrollo. El equipo proporciona al cliente una estimación que representa cuanto trabajo se puede hacer en la iteración y el cliente selecciona que es lo que se implementará durante la iteración. Por lo tanto, se mantiene el marco de trabajo anteriormente mencionado en la planificación de la versión.
	
	\item \textbf{Desarrollo:}
	El software se desarrolla para un caso de prueba. Cuando éste consigue satisfacerse se pasa al siguiente caso de prueba. Para integrar el código en el sistema principal se deben satisfacer todas las pruebas. Durante el desarrollo el equipo no debe intentar anticiparse a tareas futuras, solo centrarse en la tarea actual.
	
\end{description}
