Cómo se comentó anteriormente, éste PSR dispone de 81 restricciones. Además, no interesa cualquiera de las soluciones posibles si no que debe buscarse la solución más óptima de todas. Por lo tanto para su resolución deberá emplearme algún procedimiento lo suficientemente potente para contemplar ambos requisitos.
\subsection{Programación Lineal}
La programación lineal~\cite{Loom64} tiene como objetivo optimizar una función lineal cuyas variables están sujetas a un conjunto de restricciones lineales.
Se trata de un campo de la matemática muy efectivo para la resolución de problemas donde se desea sacar el mayor provecho de una situación.\\

Históricamente, el concepto de programación lineal debe su nombre a John Von Neumann (1947), uno de los matemáticos más importantes del siglo XX gracias a sus contribuciones en las ciencias de la computación; y a George Dantzig (1947), cuyo trabajo intentaba asignar 70 puestos de trabajo a 70 personas mediante programación lineal. Las permutaciones necesarias para la asignación óptima de dichos puestos era factorial de 70 (70!), algo enorme, pues el número de combinaciones de variables es inmenso. Curiosamente, mediante programación lineal el problema se resuelve en un momento pues el número de combinaciones se reduce en su mayor parte. La programación lineal puede ser aplicable a numerosos problemas comunes tales como:
\begin{itemize}
\item Asignación de horarios a profesores en un centro educativo para obtener la mayor productividad a la par que comodidad para profesor y alumno.
\item Distribución de elementos en almacenes de tal modo que se reduzca el costo de almacenamiento teniendo en cuenta la limitada capacidad.
\item Distribución de bienes entre compradores y consumidores de tal modo que las ganancias del intermediario sean máximas.
\end{itemize}
Cómo se puede observar, el problema de este trabajo fin de grado está muy relacionado con el último ejemplo, pues se distribuye cantidad de energía entre fuentes de entrada y fuentes de salida de manera óptima para garantizar un gasto mínimo de consumo energético.\\

Para un problema de programación lineal pueden existir varios casos en su resolución:
\begin{itemize}
\item Existe una solución óptima.
\item Existen varias soluciones óptimas.
\item No existe solución.
\item Existen soluciones infinitas.
\end{itemize}
La situación deseada es la primera, pero puede ocurrir alguno de los otro casos. Estas situaciones pueden resolverse convirtiendo las restricciones que son inecuaciones (desigualdades) en igualdades.\\

Existen varios métodos de programación lineal. El más utilizado es conocido como el \textbf{método Simplex}, pues es muy potente debido a que se basa en evaluar solo algunos puntos extremos mediante dos condiciones:
\begin{itemize}
\item \textbf{Optimalidad}. La solución inferior relativa al punto de solución actual no se tiene en cuenta.
  \item \textbf{Factibilidad}. Una vez se encuentra una solución básica factible, sólo apareceran soluciones factibles.
\end{itemize}
Otro método de programación lineal es el método de ramificación y acotamiento \textit{branch and bound}, el cuál divide el problema en varios subproblemas de programación lineal, acotamiento que permite obtener soluciones óptimas que se mejorar por cada subproblema.\\

En el ámbito de las ciencias de la computación existen librerías que permiten emplear algoritmos de programación lineal para la resolución de problemas de optimización. En este trabajo fin de grado se empleará \textbf{SciPy}, un ecosistema de librerías de código abierto con numerosas herramientas para matemáticas, ciencia e ingeniería.

\subsection{Optimización con SciPy}
Scipy~\cite{Scip} proporciona un conjunto de paquetes de computación científica para el lenguaje Python como son Numpy, Scipy, Matplotlib, iPython, SymPy y Pandas. En este caso el trabajo se centra en el módulo Scipy.optimize~\cite{SciOp}, que contiene las herramientas de Scipy para optimización. Proporciona numerosos algoritmos de optimización para uso común:
\begin{itemize}
\item Minimización sin restricciones y restringida de funciones escalares multivariadas.
\item Optimización global mediante fuerza bruta.
\item Minimización de mínimos cuadrados.
\item Minimización de funciones univariantes escalares y búsqueda de soluciones.
\item Solución de sistemas de ecuaciones multivariables con una gran cantidad de algoritmos.
\end{itemize}
Para el caso propio de este trabajo fin de grado en que se el objetivo es minimizar una función sujeta a un gran conjuntos de restricciones, lo más conveniente es hacer uso del módulo \textbf{linprog} de Scipy.optimize, específico para trabajar con programación lineal. Resuelve problemas del tipo definido en el listado~\ref{lst:linprog}, donde:
\begin{itemize}
\item A\_ub representa los coeficientes de las restricciones definidas como inecuaciones.
\item b\_ub representa las constantes del tipo de restricción inecuación.
\item A\_eq representa los coeficientes de las restricciones de igualdad, es decir, ecuaciones.
\item b\_eq representa las constantes del tipo de restricción de igualdad.
\item (lb, ub) representan los límites inferior y superior del dominio de la variable x.
\item c es la función a minimizar, dependiente de la variable x.
\end{itemize}
\begin{lstlisting}[language=Python,float=ht,numbers=none,caption={Tipo de problema aplicable a Scipy.optimize.linprog},label={lst:linprog}]
  # Minimizar:
  c @ x
  # Sujeto a:
  A_ub @ x <= b_ub
  A_eq @ x == b_eq
  lb <= x <= ub
\end{lstlisting}
El caso particular de este trabajo se adapta perfectamente a dicho modelo de problema. Pero antes de implementar el algoritmo linprog, se deben implementar cada una de las restricciones del modelo, algo complejo en este caso pues existen numerosas restricciones al tratarse de una función lineal, pues cada una de las variables involucradas en la función a minimizar~\ref{eq:funcionObjetivo} tendrá realmente 24 valores, correspondientes a las 24 horas de la simulación, y desde el punto de vista de la implementación, será tenido en cuenta como 24 variables distintas.\\

Antes de implementar cada una de las restricciones, se debe hacer una modificación en la clase Simulation. Se añaden cinco nuevos atributos a la clase:
\begin{itemize}
\item \textbf{f}: Esta variable representa la función objetivo (véase la ecuación~\ref{eq:funcionObjetivo}), expresada como una lista con los coeficientes de cada variable en la función los cuáles representan el precio del tipo de energía asociado a la variable. Al tratarse de un sumatorio de 24 iteraciones y contener 5 variables en la expresión, esta lista contendrá 120 elementos resultantes de la suma entre los 24 valores de cada una de las variables. Para dotar de valores a la lista se ha implementado la función \textit{generate\_function\_coeficients()}, que mediante 24 iteraciones concatena a la lista el valor correspondiente de cada coeficiente de variable, que se encuentran en los atributos de clase definidos en el hito 3 (Véase la representación UML de Simulation en la figura~\ref{fig:simulation}). Esta variable se corresponde con \textit{c} en el modelo de problema para Scipy Linprog del listado~\ref{lst:linprog}.
\item \textbf{A\_ub, b\_ub}: Cómo se comentó anteriormente representan las restricciones del tipo inecuación. En A\_ub se almacenan en una lista los coeficientes de las inecuaciones en una lista por restricción, de tal modo que se tiene una lista de listas (lista de dos dimensiones) del tipo: [[coeficientes restr. 1], [coeficientes restr.2], ...]. En b\_ub se tiene una lista con las constantes de las inecuaciones, por lo que el tamaño de b\_ub y A\_ub debe ser igual para que se realice el \textit{matching} de coeficientes con constantes por inecuación.
\item \textbf{A\_eq, b\_eq}: Similar a las dos listas anteriores, pero en este caso se trata de las restricciones de igualdad. Las lista tienen el mismo formato.
\end{itemize}
Éstas variables serán primordiales a la hora de ejecutar el algoritmo linprog pues de sus valores serán dependientes los resultados para cada variable. Definidas las variables que contendrán los valores de las restricciones se pasa a la implementación de dichas restricciones.
\subsection{Implementación de las restricciones del tipo 1}
La restricción 1 se corresponde con que toda la energía generada debe ser consumida (Véase la ecuación~\ref{eq:restr1}). Se trata de una restricción de igualdad, por lo que deben dotarse de valor A\_eq y b\_eq. Es una restricción lineal por lo que desde el punto de vista de la implementación se traduce en 24 restricciones, una por hora de la simulación. En el listado~\ref{lst:restr1} se muestra la función \textit{generate\_restriction\_1()} que realiza dicha tarea.\\
\begin{lstlisting}[language=Python,float=ht,caption={Restricciones del tipo 1},label={lst:restr1}]
def generate_restriction_1(self):
    for i in range(0, 24):
        restr_coef = [0]*5*24
        restr_coef[i*5] = 1
        restr_coef[i*5+1] = 1
        restr_coef[i*5+2] = 1
        restr_coef[i*5+3] = -1
        restr_coef[i*5+4] = -1
        self.A_eq.append(restr_coef)
        self.b_eq.append(self.c_int + self.c[i])
\end{lstlisting}

Primero se deben mostrar a la izquierda de la restricción las variables y a la derecha las constantes. En A\_eq se debe concatenar una lista por restricción del sumatorio que contendrá los valores 0 o 1 en función de la condición mostrada en el listado~\ref{lst:coef}
\begin{lstlisting}[numbers=none,float=ht,caption={Condición para dotar de valor los coeficientes},label={lst:coef}]
  Si la variable de esa posición aparece en la restricción
      restr_coef[posicion] = 1
  Si no
      restr_coef[posicion] = 0
\end{lstlisting}
Cómo se puede observar en el listado~\ref{lst:restr1}, por cada iteración de las 24 (correspondientes a las 24 horas de la simulación) primero se crea una lista \textbf{restr\_coef} con solo valores 0. Ésta lista dispone de 120 elementos, pues la restricción es realmente el sumatorio de 24 restricciones y existen 5 variables en la expresión (EF, ER, EB, CR y CB). Cómo resultado se obtendrá en A\_eq 24 listas de 120 elementos cada una, de los cuáles todos toman el valor 0 excepto los relativos a las posiciones de las variables que entran en juego en la restricción de esa iteración. Es primordial que se preserve el orden de ordenamiento de las variables en todas las restricciones. Deben tener el mismo orden que en la función objetivo y tomar 1 si aparecen o 0 si no (Podrán tomar el valor -1 si van precedidas de una resta en la restricción). En el caso de b\_eq, se concatenan 24 valores, uno por iteración, correspondientes l valor constante de la restricción de esa iteración.\\Obsérvese cómo se realizaría la primera iteración, correspondiente a la hora 0 de la simulación:\\

\textit{Deben dotarse con 1 $ EF_{0} $, $ ER_{0} $ y $ EB_{0} $, pues su coeficiente en la restricción es +1. Deben dotarse con -1 $ CR_{0} $ y $ CB_{0} $, pues su coeficiente es -1 en la restricción. El resto de elementos de la lista deben ser 0 (Correspondientes al resto de coeficientes de variables para i=1,2,3..). Ésta lista se concatena en A\_eq. En b\_eq se concatena el valor constante de esta restricción, que es $ c_{int} + C_{i}^{prop} $. Con esta queda implementada la restricción de tipo 1 correspondiente a la hora 0 de la simulación.}

\subsection{Implementación de las restricciones del tipo 2}
La restricción 2 hace referencia a que no se puede producir energía fotovoltaica durante la noche (Véase la ecuación~\ref{eq:restr2}). En este trabajo fin de grado se definen estos valores como los comprendidos entre las 9:30 pm y las 7:00 am. Como se observa en el listado~\ref{lst:restr2}, para generar las restricciones de este tipo, en cada iteración se inicializa la lista de 120 valores con ceros de manera análoga a las restricciones de tipo 1. Después, para determinar la hora real correspondiente de la iteración en curso, se debe sumar a la hora de inicio de la simulación el número de iteración actual. El uso del módulo de la librería estándar de Python \textit{datetime}~\cite{Dtpy} hace que sea posible manejar variables en formato hora o fecha. En las iteraciones en las que la hora actual esté comprendida en las definidas cómo horas nocturnas, el valor de la posición de EF (energía fotovoltaica) en esa iteración tomará el valor 1. Éstas listas resultantes de cada iteración se van concatenando con A\_eq, pues son restricciones de igualdad. En cuanto a b\_eq, por cada iteración se concatena un 0, pues el valor constante de esta restricción es 0 debido a que la energía fotovoltaica de noche es nula. Tras la ejecución de la función \textit{generate\_restriction\_2()} A\_eq cuenta con 24 listas mas, que son las 24 restricciones del tipo 2.
\begin{lstlisting}[language=Python,float=ht,caption={Restricciones del tipo 2},label={lst:restr2}]
def generate_restriction_2(self):
    for i in range(0, 24):
        restr_coef = [0]*5*24
        time = (self.start+dt.timedelta(hours=i)).time()
        if time >= dt.time(21, 30) or time <= dt.time(7, 00):
            restr_coef[i*5] = 1
        self.A_eq.append(restr_coef)
        self.b_eq.append(0)
\end{lstlisting}
\subsection{Implementación de las restricciones del tipo 3}
Las restricciones del tipo 3 hacen que se cumpla que la energía fotovoltaica generada no puede ser mayor que la máxima energía fotovoltaica en t, siendo t cada hora de la simulación (Véase la ecuación~\ref{eq:restr3}). Se trata de una restricción de tipo inecuación, por lo que en este caso deberán concatenarse sus valores a A\_ub y b\_ub. En la variable de clase \textit{self.max\_ef\_buffer} se dispone de una lista con los 24 valores correspondientes a la energía fotovoltaica máxima de cada hora de la simulación. Cada elemento de esta lista representa la parte constante de cada restricción de este tipo, por ello, en cuanto a b\_ub se refiere, basta con concatenar \textit{self.max\_ef\_buffer}. En el caso de la parte de variables (A\_ub), al igual que en los casos anteriores se realizan 24 iteraciones correspondientes a las 24 horas de la simulación, y en cada una de ellas, la lista de 120 elementos toma el valor 1 únicamente en la posición relativa a la energía fotovoltaica, pues es la única que entra en juego en este tipo de restricción.La lista generada en cada iteración se concatena con el resto de restricciones en A\_ub.
\begin{lstlisting}[language=Python,float=ht,caption={Restricciones del tipo 3},label={lst:restr3}]
def generate_restriction_3(self):
    for i in range(0, 24):
        restr_coef = [0]*5*24
        restr_coef[i*5] = 1
        self.A_ub.append(restr_coef)
    self.b_ub.extend(self.max_ef_buffer)
\end{lstlisting}
\subsection{Implementación de las restricciones del tipo 4}
Las restricciones de tipo 4 hacen que se cumpla que la energía obtenida de la batería no puede ser mayor que el nivel de batería actual teniendo en cuenta la profundidad máxima de descarga (Véase la ecuación~\ref{eq:restr4}). Son restricciones de tipo inecuación por lo que deben modificarse A\_ub y b\_ub. En este caso, la restricción correspondiente a la hora 0 de la simulación debe separarse de las restantes, pues en ese punto la cantidad de carga de la batería se obtiene directamente de la variable de clase que contiene el nivel inicial de batería (\textit{self.battery\_level})~\ref{eq:restr4t1} y en el resto de casos se obtiene mediante un conjunto de operaciones~\ref{eq:restr4t2}. Esto permite calcular el nivel actual de batería en la hora i a partir de la que hubo inicialmente, mediante el sumatorio de las cargas y descargas que se han realizado desde que comenzó la simulación. En el listado~\ref{lst:restr4} se puede observar la función \textit{generate\_restriction\_4()}, encargada de la implementación de las restricciones de tipo 4 comprendidas entre las horas 1 y 24 de la simulación. Por cada iteración, en la lista de coeficientes se coloca un uno en la posición relativa a EB, pues es la dependiente de esta restricción. Después, se realizan iteraciones desde 0 hasta la iteración anterior a la actual, para comprobar el nivel actual de batería, posicionando los valores 1 en EB y -1 en CB. Cuando la lista de coeficientes está completa en esa iteración, se añade a A\_ub, y en b\_ub se concatena la parte constante de este tipo de restricción, que viene a ser la diferencia entre el nivel inicial de batería y la capacidad de la misma por su profundidad de descarga.
\begin{equation}
  \label{eq:restr4t1}
  EB_{0} \leq initial\_level - capacity * depth
\end{equation}
\begin{equation}
  \label{eq:restr4t2}
  EB_{i} \leq initial\_level + \sum_{t=0}^{i-1}(-EB_{t}+CB_{t}) - capacity * depth
\end{equation}
\begin{lstlisting}[language=Python,float=ht,caption={Restricciones del tipo 4},label={lst:restr4}]
def generate_restriction_4(self):
    for i in range(1, 24):
        restr_coef = [0]*5*24
        restr_coef[i*5+2] = 1
        for j in range(0, i-1):
            restr_coef[j*5+2] = 1
            restr_coef[j*5+4] = -1
        self.A_ub.append(restr_coef)
        self.b_ub.append(self.battery_level
            -self.battery_capacity*self.discharge_depth)
\end{lstlisting}
\subsection{Implementación de las restricciones del tipo 5}
Las restricciones de tipo 5 se encargan de que el consumo para cargar la batería no pueda ser mayor que la capacidad de la misma menos el nivel restante después de t (Véase la ecuación~\ref{eq:restr5}). Las restricciones de este tipo son muy parecidas a las de tipo 4, con la diferencia de que las retricciones de tipo 4 se encargan de regular la energía que se descarga de la batería y las restricciones de tipo 5 regulan la energía que se carga a la batería. La hora 0 de la simulación debe implementarse aparte análogamente al tipo anterior, pues el nivel actual de batería se determina en función de las cargas y descargas que se han producido desde que comenzó la simulación. En este caso la restricción de la hora 0 es muy sencilla pues tras agrupar a la izquieda de la inecuación las variables y a la derecha las constantes y ordenar las variables preservando el orden de \textit{f} se obtiene la restricción~\ref{eq:restr5t1}. Para implementar esta restricción simplemente se debe dar valor de -1 a la posición relativa a $ EB_{0} $ y 1 a la posición relativa a $ CB_{0} $, para después añadir a sus respectivas listas la lista de coeficientes y el valor constante de la restricción. Para el resto de restricciones de este tipo (hora 1 a 24) se usa la función \textit{generate\_restriccion\_5()} cuya traza es similar a \textit{generate\_restriccion\_4()} exceptuando los valores que toman las posiciones relativas a las variables dependientes de la restricción.
\begin{equation}
  \label{eq:restr5t1}
  CB0 - EB0 <= capacity - initial_level
  -EB_{0} + CB_{0} \leq capacity - initial\_level
\end{equation}
\begin{lstlisting}[language=Python,float=ht,caption={Restricciones del tipo 5},label={lst:restr5}]
def generate_restriction_5(self):
    for i in range(1, 24):
        restr_coef = [0]*5*24
        restr_coef[i*5+4] = 1
        restr_coef[i*5+2] = -1
        for j in range(0, i-1):
            restr_coef[j*5+2] = -1
            restr_coef[j*5+4] = 1
        self.A_ub.append(restr_coef)
        self.b_ub.append(self.battery_capacity - self.battery_level)
\end{lstlisting}
\subsection{Generación optimizada de energía}
Una vez implementadas todas las restricciones necesarias del PSR es la hora de implementar el algoritmo linprog de Scipy. El método \textit{optimize()} de la clase Simulation se encarga de esta tarea. En ella deben llamarse todos los métodos encargados de las restricciones, para así poder contener en A\_eq, b\_eq, A\_ub y b\_ub los datos de variables y constantes necesarios. Después deben determinarse los límites que pueden tomar las variables, definidos en el hito anterior. Finalmente se efectúa el algoritmo linprog sobre todos los datos y se obtiene como respuesta un conjunto de valores que han de ser interpretados, para lo que se añaden a la clase Simulation las siguientes funciones auxiliares:
\begin{itemize}
\item \textbf{store\_result(result)}: Se encarga de almacenar en un fichero los resultados obtenidos, indicando fecha de simulación, gasto económico producido y cantidad de energía de cada fuente de entrada y salida por horas. Esta información es almacenada en un fichero llamado \textit{simulation\_fecha.txt}, que sirve como reporte de la simulación. Para obtener cada valor se itera sobre la lista de valores en bruto \textbf{\textit{res.x.to\_list()}} separando cada valor de variable en su iteración y variable correspondiente.
\item \textbf{prepare\_result(result)}: Se encarga de procesar una salida a la simulación alternativa a la anterior, pues retorna los resultados utilizando el formato json. Ésto será útil cuando se haga una petición de simulación desde el servidor y deba devolverse el resultado en este formato para poder ser procesado fácilmente. Se utiliza el método \textit{json.dumps()} para generar el objeto json a partir de un diccionario clave-valor (Véase el listado~\ref{lst:resultJson}). La función \textit{self.prepare\_hours(values)} procesa la lista de valores de variables en bruto a un diccionario clave-valor.
\begin{lstlisting}[language=Python,float=ht,caption={Función de procesamiento del resultado a formato json},label={lst:resultJson}]
def prepare_result(self, res):
    values = res.x.tolist()
    data = {
      "start" : self.start.strftime("%Y-%m-%d %H:%M:%S"),
      "end" : self.end.strftime("%Y-%m-%d %H:%M:%S"),
      "result" : res.fun,
      "hours" : self.prepare_hours(values)
    }

    return json.dumps(data)
\end{lstlisting}
\end{itemize}
Puesto que ya se cuenta con el esqueleto del proceso para llevar a cabo una simulación, se procede a realizar un caso de prueba del lunes 11 de Marzo de 2019.
\subsection{Caso de prueba: Simulación del 11 de Marzo}
Tal y cómo se explicó en el hito 1, en el Área Cliente de Endesa es posible obtener el consumo realizado por horas de un día determinado. Esto va a permitir conocer cuál fue el consumo del 11 de Marzo y, gracias al trabajo realizado con la API e-sios~\cite{Ree}, poder saber cuál fue la cuantía económica que supuso el consumo total de ese día, para así poder comparar con el gasto económico que se obtendrá de la simulación.\\

El 11 de marzo en el hogar del alumno se produjo un consumo total de 7 KWh. El precio voluntario al pequeño consumor (PVPC) por defecto de ese día obtuvo cómo valor medio 0.12 euros, por lo tanto cómo mínimo el consumo del día supuso un gasto de 0.84 euros. El fichero de texto que contiene el consumo diario de Endesa sigue el formato mostrado en el listado~\ref{lst:11marzo} (\textcopyright Endesa S.A.)\\

\begin{lstlisting}[float=ht,numbers=none,caption={Fichero de consumo por horas de Endesa},label={lst:11marzo}]
CUPS:				ESXXXXXXXXXXXXXXXXXXXX
Fecha :				11/03/2019
Fecha y hora de extracción :	23/03/2019 10:57:52
Tarifa :			No se encontró la tarifa
  Fecha 			   Hora 			    Consumo (Wh)
2019-03-11			00:00-01:00				110.0
2019-03-11			01:00-02:00				80.0
2019-03-11			02:00-03:00				166.0
2019-03-11			03:00-04:00				141.0
2019-03-11			04:00-05:00				95.0
2019-03-11			05:00-06:00				126.0
2019-03-11			06:00-07:00				186.0
2019-03-11			07:00-08:00				217.0
2019-03-11			08:00-09:00				568.0
2019-03-11			09:00-10:00				692.0
2019-03-11			10:00-11:00				280.0
2019-03-11			11:00-12:00				216.0
2019-03-11			12:00-13:00				149.0
2019-03-11			13:00-14:00				348.0
2019-03-11			14:00-15:00				677.0
2019-03-11			15:00-16:00				339.0
2019-03-11			16:00-17:00				368.0
2019-03-11			17:00-18:00				192.0
2019-03-11			18:00-19:00				202.0
2019-03-11			19:00-20:00				175.0
2019-03-11			20:00-21:00				648.0
2019-03-11			21:00-22:00				479.0
2019-03-11			22:00-23:00				318.0
2019-03-11			23:00-00:00				270.0

Total (Wh):				7042.0
\end{lstlisting}
Mediante la función \textit{read\_from\_file()} del módulo \textit{client\_consumption} se procesa el fichero de consumo, se establecen los consumos en la unidad de KWh y se retorna una lista con los 24 consumos de las 24 del día, en este caso 11 de marzo. Esta lista será el valor que toma la variable de Simulation \textit{self.c}. Debido a que aún no se dispone de interfaz ni servidor, para ejecutar la simulación se ha empleado el intérprete de comandos ipython3. Se debe crear un objeto simulación con los argumentos necesarios explicados a lo largo de este capítulo y que se pueden observar en la clase UML de Simulation (Figura~\ref{fig:simulation}). Tras instanciar en un objeto la simulación, se debe invocar el método \textit{optimize()} que implementa las restricciones y ejecuta el algoritmo de programación lineal, obteniendo como salida el contenido del listado~\ref{lst:output}.\\

\begin{lstlisting}[language=bash,float=ht,numbers=none,caption={Salida del algoritmo \textit{linprog}},label={lst:output}]
fun: 0.6187091110614097
message: 'Optimization terminated successfully.'
nit: 236
slack: array([ 8.16000000e-01,  8.16000000e-01,  8.16000000e-01,
        8.16000000e-01,  2.66000000e-01,  2.66000000e-01,
        2.66000000e-01,  2.66000000e-01,  0.00000000e+00,
                                      . . .
        5.91880000e+00,  4.68600000e+00,  1.05000000e+01,
        5.02480000e+00,  1.05000000e+01,  1.05000000e+01])
status: 0
success: True
x: array([0.00000000e+00, 1.98800000e-01, 0.00000000e+00,
       0.00000000e+00, 0.00000000e+00, 0.00000000e+00,
       1.68800000e-01, 0.00000000e+00, 0.00000000e+00,
                                     . . .
       0.00000000e+00, 0.00000000e+00, 0.00000000e+00,
       3.58800000e-01, 0.00000000e+00, 0.00000000e+00])
\end{lstlisting}
Esta información es difícil de interpretar estando fuera de contexto.
\begin{itemize}
\item \textit{fun} es el resultado que toma la función \textit{f(x)} (Función~\ref{eq:funcionObjetivo}) como resultado de asignar a las variables los valores de la optimización. Dicho valor representa el \textbf{gasto económico} producido para el día simulado. Cómo se puede observar, se trata de 0.61 euros frente a los 0.84 euros que supuso el consumo del día utilizando solamente la energía de la compañia eléctrica. Se ha producido un ahorro del 27.38\% lo cuál es bastante significativo, teniendo en cuenta que el valor medio del PVPC del día 11 de marzo fue relativamente bajo con respecto al valor que suele oscilar (0.15 euros).
\item \textit{message} devuelve \textit{feedback} sobre si la optimización ha tenido o no éxito, tomando en los campos \textit{status} y \textit{success} los valores 0 y True en caso de éxito; y los valores 1 y False en caso de error.
\item \textit{nit} hace referencia a las iteraciones realizadas sobre las variables para determinar el valor óptimo obtenido. En este caso se han realizado 236 iteraciones.
\item \textit{slack} contiene un array con los valores de las variables de holgura, donde cada variable de holgura corresponde con una de las restricciones de desigualdad.
\item \textit{x} contiene el array en bruto de valores de variables que han hecho posible la optimización. Esta información ha de ser tratada y clasificada para poder ser interpretada, por lo que mediante la función \textit{store\_result()} se ha generado un fichero de reporte de simulación que contiene el valor de cada variable (EF, ER, EB, CB, CR, C y $ C_{int} $) en cada una de las horas de la simulación. Estos valores representan la cantidad de energía en KWh que debe tomarse en cada variable para producir la minimización del gasto económico.
\end{itemize}
Véase el Anexo A pues contiene el listado~\ref{lst:simulationReport} con la información procesada de la simulación.\\Cómo se puede observar, en la hora de comienzo de la simulación (0:00), la única fuente de energía es la energía de red, pues una de las restricciones implementadas es que durante la noche la energía fotovoltaica no es posible, y la batería aún no se ha cargado para permitir su extracción. Se puede observar que en algunas horas se genera más energía de la necesaria para satisfacer el consumo (C + $ C_{int} $). Esto es debido a que mediante la API aemet~\cite{Ree} se comprueba que en ese momento el precio PVPC se encuentra relativamente bajo por lo que resulta rentable importar más energía de la necesaria y almacenarla en batería para una hora en la que los precios suban. Es el caso de lo ocurrido en la hora 2:00 a.m. por ejemplo. A partir del amacener comienza a tenerse en cuenta la energía fotovoltaica, es el caso de la hora 8:00 a.m., donde se aprovecha su potencial para importar mucha más energía de la requerida y se almacena (Se requieren 0.65 KWh pero sin embargo son generados 0.816 KWh, aprovechando la diferencia para cargar la batería). Esto permite que en horas donde por causas meteorológicas (API AEMET~\cite{Aemet}) no sea posible emplear energía fotovoltaica se tengan alternativas a la energía de red, o también durante la noche, como es el caso de la hora 11:00 p.m., cuando la energía fotovoltaica no es posible y la energía de red tiene un precio más alto, se obtenga la totalidad de la energía requerida de la batería (EB).\\

Tras esta simulación y la interpretación de resultados se han sacado las siguientes conclusiones:
\begin{itemize}
\item Se ha producido un ahorro económico en la simulación del día 11 de marzo mediante el caso propuesto con respecto al caso real, a pesar que es un día donde el PVPC se encuentra relativamente bajo, por lo que el ahorro producido en un caso de simulación de la media habría sido mucho mayor.
\item Mediante el empleo de distintas fuentes de energía se ha producido una eficiencia energética.
  \item La actuación reactiva del sistema con respecto a la información cambiante ya sea meteorológica o del mercado eléctrico permite garantizar que en cada momento la cantidad de energía en cada una de las fuentes es la cantdad óptima.
\end{itemize}
