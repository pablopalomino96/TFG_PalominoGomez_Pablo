\chapter{El primer anexo}
\label{cap:AnexoA}

En los anexos se incluirá de modo opcional material suplementario que podrá consistir en breves manuales, listados de código fuente, esquemas, planos, etc. Se recomienda que no sean excesivamente voluminosos, aunque su extensión no estará sometida a regulación por afectar esta únicamente al texto principal. 

\paragraph{Bibliografía}
Esta sección, que si se prefiere puede titularse «Referencias», incluirá un listado por orden alfabético (primer apellido del primer autor) con todas las obras en que se ha basado para la realización del TFG en las que se especificará: autor/es, título, editorial y año de publicación. Solo se incluirán en esta sección las referencias bibliográficas que hayan sido citadas en el documento. Todas las fuentes consultadas no citadas en el documento deberían incluirse en una sección opcional denominada <<Material de consulta>>, aunque preferiblemente estas deberían incluirse como referencias en notas a pie de página a lo largo del documento.

Se usará método de citación numérico con el número de la referencia empleada entre corchetes. La cita podrá incluir el número de página concreto de la referencia que desea citarse. Debe tenerse en cuenta que el uso correcto de la citación implica que debe quedar claro para el lector cuál es el texto, material o idea citado. Las obras referenciadas sin mención explícita o implícita al material concreto citado deberían considerarse material de consulta y por tanto ser agrupados como «Material de consulta» distinguiéndolas claramente de aquellas otras en las que si se recurre a la citación.

Cuando se desee incluir referencias a páginas genéricas de la Web sin mención expresa a un artículo con título y autor definido, dichas referencias podrán hacerse como notas al pie de página o como un apartado dedicado a las «Direcciones de Internet».

Todo el material ajeno deberá ser citado convenientemente sin contravenir los términos de las licencias de uso y distribución de dicho material. Esto se extiende al uso de diagramas y fotografías. El incumplimiento de la legislación vigente en materia de protección de la propiedad intelectual es responsabilidad exclusiva del autor del trabajo independientemente de la cesión de derechos que este haya convenido. De este modo será responsable legal ante cualquier acción judicial derivada del incumplimiento de los preceptos aplicables. Así mismo ante dicha circunstancia los órganos académicos se reservan el derecho a imponer al autor la sanción administrativa que se estime pertinente. 

\paragraph{Índice temático}
Este índice es opcional y se empleará como índice para encontrar los temas tratados en el trabajo. Se organizará de modo alfabético indicando el número de página(s) en el que se aborda el tema concreto señalado.
